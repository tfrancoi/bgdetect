\documentclass[11pt,a4paper]{report}
\usepackage{ifpdf}
\usepackage[utf8]{inputenc}
\usepackage[francais]{babel}
\usepackage[pdftex]{graphicx}
\usepackage{listings}
\usepackage{array}


\title{ELEC 2885 : Image processing and computer  \\ Project \\
Benoit Macq \\ Christophe Devleeschouwer \\ Antonin Descampe}
\author{Thibault François \\ Frédéric Vand der Essen}
\date{\today}



\begin{document}
	\begin{titlepage}		
		\begin{figure}[tbp]
			\begin{center}
				\includegraphics{image/logo.png}
			\end{center}
		\end{figure}
		\maketitle
	\end{titlepage}

\section{Introduction}
Dans ce projet nous essayons de séparer les éléments du background de ceux du foreground. Nous définissons le background comme les éléments du décors
qui ne bouge pas, par conséquent notre technique ne fonctionnera qu'avec des images prises d'une caméra fixe. 

\section{Features extractions}


\section{Random Forest classification}

\section{Parameter tuning}


\section{Result}

\section{Conclusion}
	
	
	







\end{document}
