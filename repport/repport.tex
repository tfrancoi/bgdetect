\documentclass[8pt,a4paper]{report}
\usepackage{ifpdf}
\usepackage[utf8]{inputenc}
\usepackage[francais]{babel}
\usepackage[pdftex]{graphicx}
\usepackage{listings}
\usepackage{multicol}
\usepackage{array}


\title{ELEC 2885 : Image processing and computer  \\ Project \\
Benoit Macq \\ Christophe Devleeschouwer \\ Antonin Descampe}
\author{Thibault François \\ Frédéric Vand der Essen}
\date{\today}



\begin{document}
	\begin{titlepage}		
		\begin{figure}[tbp]
			\begin{center}
				\includegraphics{image/logo.png}
			\end{center}
		\end{figure}
		\maketitle
	\end{titlepage}

\begin{multicols}{2}
\section{Introduction}
Dans ce projet nous essayons de séparer les éléments du background de ceux du foreground. Nous définissons le background comme les éléments du décors
qui ne bougent pas, par conséquent notre technique ne fonctionnera qu'avec des images prises d'une caméra fixe. 

\section{Features extractions}
	L'approche par arbres aléatoire permet de classer les pixels selon plusieurs critères. Ils nous faut donc dans un premier temps définir ces critères. 
	La première feature est l'intensité du pixel, c'est à dire la moyenne des trois composants RGB. 
	Nous avons extrait deux autres features le gradient de l'intensité selon l'horizontale et selon la verticale \cite{gradient} 
	grâce à l'opérateur de convolution de Sobel \cite{sobel}. L'intérèt du gradiant étant que celui ci reste constant si l'intensité lumineuse globale change,
	et que celui ci capture la forme de l'objet. 

	D'autres features possibles sont la teinte et la saturation, qui sont elles aussi moins sensibles aux changements d'éclairage de la scène. \cite{hsv} 

\section{Random Forest classification}

	Nous avons donc en entrée une suite d'images dont chaque pixel est défini par un vecteur de features.
	Si on regarde l'évolution temporelle des features d'un pixel, on remarque qu'on obtient des valeurs très
	semblables lorsqu'il fait partie du bakcground. Si on fait un histogramme des features on remarque un pic
	pour les pixels appartenant au bakckground, et d'autres pics plus petits pour les pixels appartenant aux
	éléments transients. 

	Pour classer un pixel, on le place dans l'histogramme et on regarde s'il se trouve dans le pic principal. 
	Habituellement, cet histogramme est modélisé par une mixture de gaussienne. Nous allons ici modéliser cet
	histogramme par un arbre binaire totalement aléatoire. L'avantage étant que l'arbre binaire peut traiter
	plus facilement un grand nombre de features.

	L'arbre binaire totalement aléatoire sépare à chaque noeud l'espace des features, avec une feature et une
	valeur de séparation choisies aléatoirement selon une distribution uniforme. Les feuilles de l'arbre 
	contiennent la liste des pixels correspondant. L'arbre est construit dynamiquement à chaque ajout de pixel,
	en respectant un critère de profondeur maximale.

	On s'attend donc à ce que les pics de l'histogramme se traduisent en feuilles remplies de nombreux samples.

	Pour déterminer si un pixel appartient au background, on retrouve la feuille de l'arbre correspondant et on
	détermine avec une heuristique si cette feuille correspond à un pic du background ou non. Pour améliorer
	la précision de ce processus très aléatoire on crée un grand nombre d'arbre par pixel que l'on fait ensuite
	voter.

	\subsection{Heuristique}
		\subsubsection{Heuristique1}
			Pour déterminer si une feuille appartient à un pic, on calcule le nombre moyen de pixel par
			feuille, et on regarde si la feuille correspondante a une taille supérieure ou inférieure
			à cette moyenne. On peut paramétriser cette heuristique en multipliant préalablement la moyenne
			par un facteur.

			Cette heuristique a plusieurs inconvénients. Premièrement une feuille peut avoir un grand nombre
			de pixels car elle représente un grande partie de l'espace des features. Ensuite si le nombre de
			feuilles est trop grand par rapport au nombre de pixels, les feuilles ne contiendront plus suffisemment
			de pixels pour pouvoir faire la différence entre les pics et les creux.
		\subsubsection{Heuristique2}
			Cette heuristique travaille sur la densité des feuilles, c'est à dire le nombre de pixels divisé par le
			volume de l'espace de feature correspondant à la feuille.  Comme précédemment on calcule la densité 
			moyenne des feuilles. Si la densité d'une feuille est supérieure à la moyenne, on la considère comme faisant
			partie du background.

			Comme chaque feuille contient au moins un pixel, la densité d'une feuille ne contenant qu'un pixel n'est pas
			représentative. On remarque que les densités de ces feuilles peuvent varier de trois ordres de grandeur, bien
			au delà des pics réels. Elles sont donc ignorées dans le calcul de la moyenne, et automatiquement classées
			comme faisant partie du foreground. 

			Cette heuristique est donc également vulnérable aux arbres trop profond et ayant trop de feuilles. 

	\subsection{Améliorations}
		Comme vu précédemment, augmenter la profondeur de l'arbre pose problème car cela crée un trop grand nombre de feuilles, 
		et empèche ainsi les heuristiques de bien fonctionner. Cela pose problème car plus on augmente le nombre de features,
		plus il faut que l'arbre soit profond afin d'avoir la chance de discriminer sur chaque feature. 

		Pour contourner ce problème, on va déterminer un critère de qualité de l'arbre, et ensuite ne faire de décision que sur
		les arbres de bonne qualité. 
		\subsubsection{Critère de qualité}
			Le premier critère que nous avons imaginé est de calculer une sorte d'entropie sur la densité de feuilles : 
			Soit $d_{i}$ la densité de la feuille $i$, les densités étant pondérées pour valoir $1$, on calcule le critère 
			
			\[ C = -\sum_{i} d_{i}\log{D_{i}} \]

			Des grands $C$ représentent un grand nombre de feuilles contenant un petit nombre de pixels, Au contraire, un petit $C$
			représente un petit nombre de feuilles avec un grand nombre de pixels, et donc un meilleur groupement des pixels
			similaires. 

			Nous générons donc pour chaque pixel un grand nombre d'arbres et décidons sur les arbres ayant le $C$ le plus petit.

			Il faut cependant remarquer que cette amélioration se fait au prix d'une dégradation importante des performances,
			le calcul de l'entropie et le tri des arbres n'étant pas gratuits.
		\subsubsection{Fenêtre de décision}
			Jusqu'a présent, l'algorithme utilise lors de la décision les pixels appartenant à tous les temps. Cela veut dire qu'il
			utilise les temps futurs, qui pourraient ne pas être disponibles si l'algorithme doit s'exécuter en temps réel. Cela
			pause aussi des problèmes pour des objets qui bougent et qui s'arrètent un long laps de temps. S'ils restent plus
			de 50\% du temps total au même endroit il devient difficile de savoir si l'objet appartient au background ou non.

			Il est donc possible pendant la décision de ne considérer que les pixels appartenant au passé pour un certain nombre de 
			frames, cependant si on utilise un critère de qualité d'arbre, il faut recalculer et retrier à chaque frame.

\section{Tuning des paramètres}

	Il y a de nombreux paramètres à mettre au point, et ils sont trop nombreux pour pouvoir tous les comparer, ou pour pouvoir trouver la combinaison optimale.

	
	Nous avons donc travaillé sur chaque paramètre indépendemment des autres afin de voir si ils ont une influence positive ou négative sur le résultat. 
	\subsection{Features}
		Afin de déterminer la pertinence des features, nous avons fait une première analyse avec l'intensité seule, et ensuite l'intensité 
		plus les deux gradients.
		On peut remarquer que l'ajout du gradient donne de bien meilleurs résultats.
	\subsection{Nombre d'arbres}
	Augmenter le nombre d'arbres augmente évidemment la précision de l'analyse, jusqu'à un certain point. Un compromis entre le temps de calcul et la
	précision doit être trouvé au cas par cas. 

	\subsection{Profondeur des arbres}
	La profondeur des arbres est encore un paramètre à déterminer, nous aurions pu décider d'arrêter la profondeur de l'arbre lorsque la découpe 
	des bornes serait suffissante mais il aurait quand même fallut déterminer ``la découpe suffisante''. Nous avons donc choisi de 
	définir nous même la profondeur de l'arbre. Une trop grande profondeur n'est pas forcément souhaitable pour les résultats et 
	est toujours synonyme de plus de calcul, trop peu de profondeur ne permet pas non plus une bonne classification. 
	Par défault nous utilisons une profondeur de 8 pour 3 features. La profondeur des arbres idéale est liée au nombre de features.

	\subsection{Choix de l'heuristique de classification}
	Nous avons présenté deux heuristiques de classification. Il s'avère que la seconde est nettement préférable à la première. 

	\subsection{Choix d'arbres de qualité}
	Choisir des arbres de qualité diminue légèrement le bruit présent dans l'image, mais semble diminuer également la confiance des votes
	de classification des pixels en tant que foreground. Cela reste un ajout intéressant. 

	\subsection{Système de vote}
	On peut se demander à partir de quel pourcentage des votes doit on classifier en tant que background. Nous avons trouvé que le bon
	seuil est 80\% des arbres votant en tant que bakcground.

\section{Resultat}

Les résultats sont des résultats visuels, plus le pixel est noir moins il a de probabilité d'être dans le background. Si on reconstitue un film avec les images générer on doit voir des taches noir qui se déplace sur un fond blanc. Pour rendre plus net le résultat, nous appliquons une fonction qui renvoie foreground (noir) ou background (blanc) en fonction de la probabilité du point d'appartenir au background et un seuil qu'il faut déterminer (par exemple 0.8). Pour déterminer quelle est la valeur otpimale pour un paramètre nous travaillons sans ce filtre qui gomme le bruit. 

\subsection{Heuristique}
Le choix de l'heuristique est évident quand on regarde les videos \textit{heuristique 1} et \textit{heuristique 2}. L'heuristique 2 est génère beaucoup moins de bruit et le foreground reste tout aussi bien détecté. 

\subsection{La profondeur de l'arbre}

\section{Conclusion}
Nous avons réussit à développer un algorithme de reconnaissance de background basé sur des arbres construit aléatoirement
	
	
\section{Bibliographie}	
\begin{thebibliography}{2}
   \bibitem{gradient} http://en.wikipedia.org/wiki/Image\_gradient
   \bibitem{sobel} http://en.wikipedia.org/wiki/Sobel\_operator
   \bibitem{hsv} http://en.wikipedia.org/wiki/HSL\_and\_HSV  
\end{thebibliography}

\end{multicols}



\end{document}
